\documentclass[a4paper]{article}

\usepackage[english]{babel}
\usepackage[utf8]{inputenc}
\usepackage{amsmath}
\usepackage{graphicx}
\usepackage[colorinlistoftodos]{todonotes}

\title{Causal Discovery in Complex Systems}

\author{Your Name(s)}

\date{\today}

\begin{document}
\maketitle

\begin{abstract}
[Provide a concise summary of the problem, method, data, and findings.]
\end{abstract}

\section{Introduction}
\label{sec:introduction}

[Introduce the problem of causal discovery, its importance in science and machine learning, key challenges, and an overview of your contributions.]

\section{Related Work}
\label{sec:related-work}

[Summarize existing methods in causal discovery, such as constraint-based, score-based, and functional approaches. Mention key models like PC, FCI, LiNGAM, GES, and NOTEARS.]

\section{Background and Theory}
\label{sec:background}

\subsection{Causal Graphical Models}
[Define directed acyclic graphs (DAGs), d-separation, and the causal Markov condition.]

\subsection{Identifiability and Assumptions}
[Discuss assumptions like faithfulness, causal sufficiency, and additive noise models.]

\subsection{Causal Discovery Algorithms}
[Briefly describe the classes of causal discovery algorithms relevant to your paper.]

\section{Methodology}
\label{sec:methodology}

\subsection{Problem Formulation}
[Define the causal discovery problem and your setting (e.g., observational/interventional, linear/nonlinear).]

\subsection{Proposed Method}
[Describe your approach, including theoretical justification and algorithmic details.]

\subsection{Implementation Details}
[Note implementation choices, optimization, libraries, or toolkits used.]

\section{Experiments}
\label{sec:experiments}

\subsection{Datasets}
[Describe synthetic or real-world datasets used for benchmarking.]

\subsection{Experimental Setup}
[Detail your evaluation protocol, baselines, metrics (e.g., SHD, AUC), and parameters.]

\subsection{Results and Analysis}
[Present results as tables or figures. Include comparisons and ablations if applicable.]

\section{Discussion}
\label{sec:discussion}

[Interpret the results, limitations, practical implications, and how this method advances the field.]

\section{Conclusion}
\label{sec:conclusion}

[Summarize the main contributions, results, and potential future work.]

\begin{thebibliography}{9}
\bibitem{pearl2009causality}
  Judea Pearl,
  \emph{Causality: Models, Reasoning and Inference}.
  Cambridge University Press, 2009.

\bibitem{spirtes2000causation}
  Peter Spirtes, Clark Glymour, and Richard Scheines,
  \emph{Causation, Prediction, and Search}.
  MIT Press, 2000.

\bibitem{zheng2018dags}
  Xun Zheng et al.,
  \emph{DAGs with NO TEARS: Continuous Optimization for Structure Learning}.
  NeurIPS, 2018.
\end{thebibliography}

\end{document}
