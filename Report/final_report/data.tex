\documentclass[main.tex]{subfiles}
\begin{document}

\section{Data Description and Preparation}

\subsection{Source, Context. and variable Selection}
We use the 2018 Health Behaviour in School-aged Children (HBSC) dataset, a cross-national survey supported by the WHO, which gathers health-related information from adolescents aged 11, 13, and 15. This dataset encompasses over 120 variables and covers aspects such as well-being, social relationships, behaviours, and demographic characteristics across more than 40 countries. The dataset comprises 120 variables.

These variables offer in-depth insights into individuals' health, well-being, and lifestyles. Furthermore, they include indicators related to family structure, affluence, and socio-economic status. For our research, the most critical variables are those measuring the frequency of bullying and how well individuals perceive their social integration within groups, including school peers and friends, the expression of their emotions, and their overall life satisfaction(See Table \ref{tab:hbsc_variables_grouped}  - key categories are in bold).
\begin{table}[!ht]
  \centering
  \begin{tabular}{|p{40mm}|p{100mm}|}
  \toprule
  \textbf{Category}&\textbf{Variables}\\
  \toprule
    Demographics \& Metadata & HBSC, seqno\_int, cluster, countryno, region, id1, id2, id3, id4, weight, adm, month, year, age, agecat, sex, grade, monthbirth, yearbirth \\ \hline
    Family Affluence Scale & fasfamcar, fasbedroom, fascomputers, fasbathroom, fasdishwash, fasholidays, IRFAS, IRRELFAS\_LMH \\ \hline
    Health \& Well-being & health, lifesat, MBMI, IOTF4, oweight\_who \\ \hline
    & headache, stomachache, backache, feellow, irritable, nervous, sleepdificulty, dizzy, thinkbody \\ \hline
    Health Behaviors & physact60, timeexe, breakfastwd, breakfastwe, fruits\_2, vegetables\_2, sweets\_2, softdrinks\_2, fmeal, toothbr, smokltm, smok30d\_2, alcltm, alc30d\_2, drunkltm, drunk30d, cannabisltm\_2, cannabis30d\_2 \\ \hline
    Body Measures & bodyweight, bodyheight \\ \hline
    \textbf{School Experience }& likeschool, schoolpressure, studtogether, studhelpful, studaccept, teacheraccept, teachercare, teachertrust \\ \hline
    \textbf{Violence and Bullying} & bulliedothers, beenbullied, cbulliedothers, cbeenbullied, fight12m, injured12m \\ \hline
    \textbf{Peer Support} & friendhelp, friendcounton, friendshare, friendtalk \\ \hline
    \textbf{Emotional Communication Preferences} & emconlfreq1, emconlfreq2, emconlfreq3, emconlfreq4, emconlpref1, emconlpref2, emconlpref3, emcsocmed1, emcsocmed2, emcsocmed3, emcsocmed4, emcsocmed5, emcsocmed6, emcsocmed7, emcsocmed8, emcsocmed9 \\ \hline
    Sexual Health & hadsex, agesex, contraceptcondom, contraceptpill \\ \hline
    Migration Background & countryborn, countrybornmo, countrybornfa \\ \hline
    Household Composition & motherhome1, fatherhome1, stepmohome1, stepfahome1, fosterhome1, elsehome1\_2 \\ \hline
    Parental Employment & employfa, employmo, employnotfa, employnotmo \\ \hline
    \textbf{Parent–Child Communication} & talkfather, talkstepfa, talkmother, talkstepmo \\ \hline
    \textbf{Family Support} & famhelp, famsup, famtalk, famdec \\ \hline
  \end{tabular}
  \caption{120 fields from the HBSC dataset categorised according to the content of their corresponding survey questions. }
  \label{tab:hbsc_variables_grouped}
\end{table}
% In the following sections, we select only a subset that brings the most information to our analysis.
The dataset includes two computed indicators:
\begin{itemize}
 \item \texttt{IRFAS} – Family Affluence Scale III (continuous score)
 \item \texttt{IRRELFAS} – Relative family affluence category (categorical: low, medium, high)
\end{itemize}
 
Variables such as IRFAS and IRRELFAS\_LMH are determined by a set of indicators that reflect family affluence. Table \ref{tab:FAS_variables} outlines the variables used to calculate the Family Affluence Scale III (FAS III). Despite a noticeable gap of nearly 6 points in the average weighted FAS III score for Luxembourg and Armenia (Figure \ref{fig:IRFAS_histogram}), the weighted mode of Relative FAS remains consistent across all regions at a medium level. This prompted us to investigate the relationship between the continuous and categorical versions of the score. Publications[\cite{McCormack2011}] that FAS is calculated using the following scheme: Responses to individual items were summed to derive a total FAS score and then categorized into three groups (low FAS score = 0 to 4; medium FAS score = 5 or 6; or high FAS score = 7 or 8). However, as one can observe in the table \ref{table IRFAS_IRRELFAS_uniq}, it does not accurately reflect the real data setup. Therefore, we chose to disregard the categorical version of the score and treated the continuous score as a categorical value, as it is an integer that varies from 0 to 13. The value is set to None if a respondent has not answered any of the component questions (Table \ref{tab:FAS_variables}). We deem FAS III important for our further analysis as it reflects the socio-economic background of an individual.
\begin{table}[ht]
\centering
\begin{tabular}{|l|l@{}|}
\toprule
\textbf{Variable Name} & \textbf{Description} \\
\midrule
\texttt{fasfamcar}   & Does your family own a car, van or truck? \\
\texttt{fasbedroom}  & Do you have your own bedroom for yourself? \\
\texttt{fascompu}    & Number of computers (PCs, laptops, tablets) in the household \\
\texttt{fasbathr}    & Number of bathrooms in the home (with bath or shower) \\
\texttt{fasdishw}    & Does your family have a dishwasher? \\
\texttt{fasholid}    & How many times did your family travel abroad for holidays last year? \\
\bottomrule
\end{tabular}
\caption{Variable descriptions for Family Affluence Scale (FAS) items}
\label{tab:FAS_variables}
\end{table}


 
\begin{figure}[htbp]
    \centering
    \includegraphics[width=0.8\textwidth]{Report/final_report/pictures/IRFAS_histogram.png}
    \caption{Average IRFAS score by region. The black number indicates the IRFAS score, while the grey number (n=) represents the sample size for each region.}
    \label{fig:IRFAS_histogram}
\end{figure}

\begin{longtable}{rl}
\caption{Relative Family Affluence Score(categorical) and corresponding discrete FAS III} \label{tab:IRFAS_IRRELFAS_uniq} \\
\toprule
IRRELFAS_LMH & IRFAS \\
 & unique \\
\midrule
\endfirsthead
\caption[]{Relative Family Affluence Score(categorical) and corresponding discrete FAS III} \\
\toprule
IRRELFAS_LMH & IRFAS \\
 & unique \\
\midrule
\endhead
\midrule
\multicolumn{2}{r}{Continued on next page} \\
\midrule
\endfoot
\bottomrule
\endlastfoot
1.000 & [1. 2. 0. 5. 6. 3. 4. 7. 8. 9.] \\
2.000 & [ 6.  4.  5.  8.  9.  3.  7. 11. 10. 12.  2.  1.] \\
3.000 & [11. 10. 12. 13.  9.  7.  8.  6.] \\
\end{longtable}
As a part of this report. We have decided to focus on Ukraine data, denoted as region UA in the dataset, or country number 804000. This decision is informed by personal experience and cultural context when interpreting the results of Causal Discovery algorithms. 

% \begin{longtable}{lllllllllllllllll}
\caption{Summary table of variables from the HBSC dataset, where each variable is assigned to a category depending on the data it represents} \label{tab:hbsc_variables_groups} \\
\toprule
Category & Demographics & Metadata & Family Affluence Scale & Health & Well-being & Somatic Symptoms & Health Behaviors & Body Measures & School Experience & Violence and Bullying & Peer Support & Emotional Communication Preferences & Sexual Health & Migration Background & Household Composition & Parental Employment & Parent�Child Communication & Family Support \\
\midrule
\endfirsthead
\caption[]{Summary table of variables from the HBSC dataset, where each variable is assigned to a category depending on the data it represents} \\
\toprule
Category & Demographics & Metadata & Family Affluence Scale & Health & Well-being & Somatic Symptoms & Health Behaviors & Body Measures & School Experience & Violence and Bullying & Peer Support & Emotional Communication Preferences & Sexual Health & Migration Background & Household Composition & Parental Employment & Parent�Child Communication & Family Support \\
\midrule
\endhead
\midrule
\multicolumn{17}{r}{Continued on next page} \\
\midrule
\endfoot
\bottomrule
\endlastfoot
Variables & HBSC, seqno_int, cluster, countryno, region, id1, id2, id3, id4, weight, adm, month, year, age, agecat, sex, grade, monthbirth, yearbirth & fasfamcar, fasbedroom, fascomputers, fasbathroom, fasdishwash, fasholidays, IRFAS, IRRELFAS_LMH & health, lifesat, MBMI, IOTF4, oweight_who & headache, stomachache, backache, feellow, irritable, nervous, sleepdificulty, dizzy, thinkbody & physact60, timeexe, breakfastwd, breakfastwe, fruits_2, vegetables_2, sweets_2, softdrinks_2, fmeal, toothbr, smokltm, smok30d_2, alcltm, alc30d_2, drunkltm, drunk30d, cannabisltm_2, cannabis30d_2 & bodyweight, bodyheight & likeschool, schoolpressure, studtogether, studhelpful, studaccept, teacheraccept, teachercare, teachertrust & bulliedothers, beenbullied, cbulliedothers, cbeenbullied, fight12m, injured12m & friendhelp, friendcounton, friendshare, friendtalk & emconlfreq1, emconlfreq2, emconlfreq3, emconlfreq4, emconlpref1, emconlpref2, emconlpref3, emcsocmed1, emcsocmed2, emcsocmed3, emcsocmed4, emcsocmed5, emcsocmed6, emcsocmed7, emcsocmed8, emcsocmed9 & hadsex, agesex, contraceptcondom, contraceptpill & countryborn, countrybornmo, countrybornfa & motherhome1, fatherhome1, stepmohome1, stepfahome1, fosterhome1, elsehome1_2 & employfa, employmo, employnotfa, employnotmo & talkfather, talkstepfa, talkmother, talkstepmo & famhelp, famsup, famtalk, famdec \\
\end{longtable}


\subsection{Missing Data}
In the appendix, the number of missing values, coded as sysmiss and represented as 'Nones' in the dataset is shown. Besides that, there are also more types of missingness observed in data, such as "Missing due to skip pattern"(99) or "Missing due to inconsistent answer"(-99).

% \begin{longtable}{lrrrrrrr}
\caption{Crosstabulation of responses to how easy it is for respondents to talk to their father and stepfather about things that really bother them. Values represent the number of respondents selecting each pair of response categories. Responses range from 1 (Very easy) to 4 (Very difficult), reflecting increasing difficulty. The values 0 and 5 represent non-substantive responses: 0 indicates a missing response, and 5 corresponds to "Dont have or see" the person in question.} \label{tab:father_stepfather_support_pivot} \\
\toprule
talkstepfa & 0 & 1 & 2 & 3 & 4 & 5 & Total \\
talkfather &  &  &  &  &  &  &  \\
\midrule
\endfirsthead
\caption[]{Crosstabulation of responses to how easy it is for respondents to talk to their father and stepfather about things that really bother them. Values represent the number of respondents selecting each pair of response categories. Responses range from 1 (Very easy) to 4 (Very difficult), reflecting increasing difficulty. The values 0 and 5 represent non-substantive responses: 0 indicates a missing response, and 5 corresponds to "Dont have or see" the person in question.} \\
\toprule
talkstepfa & 0 & 1 & 2 & 3 & 4 & 5 & Total \\
talkfather &  &  &  &  &  &  &  \\
\midrule
\endhead
\midrule
\multicolumn{8}{r}{Continued on next page} \\
\midrule
\endfoot
\bottomrule
\endlastfoot
0 & 2712 & 267 & 389 & 252 & 146 & 1487 & 5253 \\
1 & 20628 & 4954 & 2997 & 2028 & 1247 & 41923 & 73777 \\
2 & 18105 & 944 & 4816 & 3012 & 1582 & 49887 & 78346 \\
3 & 7975 & 503 & 1438 & 3269 & 1627 & 25477 & 40289 \\
4 & 3322 & 302 & 599 & 679 & 2714 & 11394 & 19010 \\
5 & 2045 & 921 & 1661 & 1087 & 1311 & 8159 & 15184 \\
Total & 54787 & 7891 & 11900 & 10327 & 8627 & 138327 & 231859 \\
\end{longtable}




% \begin{longtable}{rrrrrrr}
\caption{Crosstabulation of responses to how easy it is for respondents to talk to their mother and stepmother about things that really bother them. Values represent the number of respondents selecting each pair of response categories. Responses range from 1 (Very easy) to 4 (Very difficult), reflecting increasing difficulty. The values 0 and 5 represent non-substantive responses: 0 indicates a missing response, and 5 corresponds to "Dont have or see" the person in question.} \label{tab:mother_stepmother_support_pivot} \\
\toprule
0 & 1 & 2 & 3 & 4 & 5 & Total \\
\midrule
\endfirsthead
\caption[]{Crosstabulation of responses to how easy it is for respondents to talk to their mother and stepmother about things that really bother them. Values represent the number of respondents selecting each pair of response categories. Responses range from 1 (Very easy) to 4 (Very difficult), reflecting increasing difficulty. The values 0 and 5 represent non-substantive responses: 0 indicates a missing response, and 5 corresponds to "Dont have or see" the person in question.} \\
\toprule
0 & 1 & 2 & 3 & 4 & 5 & Total \\
\midrule
\endhead
\midrule
\multicolumn{7}{r}{Continued on next page} \\
\midrule
\endfoot
\bottomrule
\endlastfoot
1749 & 188 & 226 & 149 & 69 & 902 & 3283 \\
30505 & 5116 & 3667 & 3092 & 2487 & 71032 & 115899 \\
17244 & 671 & 3755 & 2570 & 1926 & 48747 & 74913 \\
4830 & 340 & 608 & 1913 & 1107 & 15244 & 24042 \\
1728 & 181 & 250 & 217 & 1622 & 5278 & 9276 \\
712 & 368 & 358 & 200 & 203 & 2605 & 4446 \\
56768 & 6864 & 8864 & 8141 & 7414 & 143808 & 231859 \\
\end{longtable}



\subsection{Variables Selected}

\subsection{Data Issues and Preprocessing}
\begin{itemize}
  \item \textbf{Missing data}: [How you handled NA values - deletion, imputation, etc.]
  \item \textbf{Data types}: [How categorical variables were encoded - label encoding, discretization]
  \item \textbf{Sample size}: [Number of observations after cleaning]
\end{itemize}

% \subsection{Scientific Question}
% Relationship of social integration of a child and aggression levels channeled through bullying and participation in fights.

% \begin{figure}[h]
%   \centering
%   \begin{subfigure}[t]{0.48\textwidth}
%     \centering
%     \includegraphics[width=\textwidth]{Report/pics/cycle of abuse.png}
%     \caption{Inspirational diagram for forming a scientific question.}
%   \end{subfigure}
%   \hfill
%   \begin{subfigure}[t]{0.48\textwidth}
%     \centering
%     \includegraphics[width=\textwidth]{Report/pics/scheme for scientific question.png}
%     \caption{Conceptual scheme illustrating the scientific question.}
%   \end{subfigure}
%   \caption{The first figure serves as inspiration, while the second is a vague schematic formulation of our scientific question.}
% \end{figure}
\end{document}