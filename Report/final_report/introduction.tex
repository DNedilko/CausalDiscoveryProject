\section{Introduction}
Causal Discovery aims to uncover causal relationships among multiple variables in a data-driven manner. This approach seeks to understand causality in an entire system of variables, which causal graphs can visualise. A graph $\mathcal{G} = (V, \mathcal{E})$, where $V$ are vertices or nodes and $\mathcal{E}$ are edges that connect these nodes. Nodes correspond to the variables in the dataset, while edges represent a connection between these variables. There can be directed, undirected, bidirected, or no edge between two nodes. A graph that allows only one-way directed edges and contains no cycles is called a Directed Acyclic Graph (DAG). In a DAG, an edge direction represents a belief that there is a direct causal relation between two variables. An absence of an edge represents that there is no direct relation. For example, an edge $X \rightarrow Y$ means that variable X causes Y. It is not always possible to learn a concrete DAG from the data, but one can find a Markov Equivalence Class (MEC) of a $\mathcal{G}$ that is represented by a CPDAG or Completed Partial DAG. CPDAG can contain both one-way oriented and unoriented edges. The existence of an edge in a CPDAG means that there is a connection between two variables, but we cannot claim causal direction due to a lack of information learned from data. A set of Markov equivalent graphs forms the Markov Equivalence Class. Two graphs are Markov Equivalent if and only if they have the same skeleton and v-structure. The skeleton of a graph $\mathcal{G}$ is the set of edges $\mathcal{E}$.

[Introduce causal discovery and its relevance in analyzing real-world observational data such as HBSC. Briefly mention the motivation for exploring adolescent health behaviors using a causal lens, and the expected outcomes of your analysis.]
