\section{Introduction}
Bullying and its variations are serious problems that are on the rise among adolescents and children \cite{Zhu2021} and are recognised as a significant public health issue. They are said to impact young individuals, affecting their mental health and prompting harmful or risky behaviours. Bullying has existed for generations, while cyberbullying is a more recent phenomenon arising from the widespread use of digital devices. Some argue, perhaps too readily, that an individual's digital presence incites malicious behaviour. These notions stem primarily from apprehension towards the largely unregulated virtual space. Current moderation of websites and apps is inadequate to safeguard against or prevent violent outbursts.

However, is it wise to view the digital world as the root of the problem concerning increasing aggression levels in children? We suggest that the issue should be considered in a broader context, examining how environmental factors influence the development of bullying tendencies. 
In this report, we assume that the internal motivation for cyberbullying and bullying is not significantly different. We propose that cyberbullying has a lower threshold for entry, driven by factors such as anonymity, accessibility, and the absence of direct social feedback in digital spaces. These elements make it easier for individuals, particularly adolescents, to engage in bullying behaviours online than in traditional, face-to-face settings. 

To further explore additional factors shaping the mindset of bullies, we will apply causal discovery methods to data from the 2018 Health Behaviour in School-aged Children(HBSC) self-report questionnaire. We will focus our analysis on Ukraine, as our personal life experience and cultural familiarity with the region will enable a more nuanced interpretation of the results. 

Causal Discovery aims to uncover causal relationships among multiple variables in a data-driven manner. This approach seeks to understand causality in an entire system of variables, which causal graphs can visualise. A graph $\mathcal{G} = (V, \mathcal{E})$, where $V$ are vertices or nodes and $\mathcal{E}$ are edges that connect these nodes. Nodes correspond to the variables in the dataset, while edges represent a connection between these variables. There can be directed, undirected, bidirected, or no edge between two nodes. A graph that allows only one-way directed edges and contains no cycles is called a Directed Acyclic Graph (DAG). In a DAG, an edge direction represents a belief that there is a direct causal relation between two variables. An absence of an edge represents that there is no direct relation. For example, an edge $X \rightarrow Y$ means that variable X causes Y.

It is not always possible to learn a concrete DAG from the data, but one can find a Markov Equivalence Class (MEC) of a $\mathcal{G}$ that is represented by a CPDAG or Completed Partial DAG. CPDAG can contain both one-way oriented and unoriented edges. The existence of an edge in a CPDAG indicates a connection between two variables, yet we cannot assert a causal direction due to insufficient information obtained from data. Various algorithms can be employed to uncover these relationships. In this particular report, we will focus on constraint-based methods that utilise conditional independence tests to understand both the presence and direction of a causal path. This report aims to obtain a CPDAG or a partial ancestral graph (PAG) that describes causal relations between the environment and bullying in teenagers aged 11 to 16 years, based on HBSC data, and to interpret and analyse the results of a selected algorithm while considering possible improvements, advantages, and drawbacks of the chosen approach. 

% A set of Markov equivalent graphs forms the Markov Equivalence Class. Two graphs are Markov Equivalent if and only if they have the same skeleton and v-structure. The skeleton of a graph $\mathcal{G}$ is the set of edges $\mathcal{E}$.

% [Introduce causal discovery and its relevance in analyzing real-world observational data such as HBSC. Briefly mention the motivation for exploring adolescent health behaviors using a causal lens, and the expected outcomes of your analysis.]
