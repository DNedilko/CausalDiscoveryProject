\documentclass[main.tex]{subfiles}
\begin{document}

\section{Introduction}
Bullying and its variations are serious problems that are on the rise among adolescents and children \cite{Zhu2021} and are recognised as a significant public health issue. They are said to impact young individuals, affecting their mental health and prompting harmful or risky behaviours. Bullying has existed for generations, while cyberbullying is a more recent phenomenon arising from the widespread use of digital devices. Some argue, perhaps too readily, that an individual's digital presence incites malicious behaviour. These notions stem primarily from apprehension towards the largely unregulated virtual space. Current moderation of websites and apps is inadequate to safeguard against or prevent violent outbursts.

However, is it wise to view the digital world as the root of the problem concerning increasing aggression levels in children? The issue should be considered in a broader context, examining how environmental factors influence the development of bullying tendencies. 
In this report, we assume that the internal motivation for cyberbullying and bullying is not significantly different. We propose that cyberbullying has a lower threshold for entry, driven by factors such as anonymity, accessibility, and the absence of direct social feedback in digital spaces. These elements make it easier for individuals, particularly adolescents, to engage in bullying behaviours online than in traditional, face-to-face settings. 

To further explore additional factors shaping the mindset of bullies, we will apply causal discovery methods to data from the 2018 Health Behaviour in School-aged Children(HBSC) self-report questionnaire. We will focus our analysis on data from  Ukraine. 

Causal Discovery aims to uncover causal relationships among multiple variables in a data-driven manner. This approach seeks to understand causality within an entire system of variables, which causal graphs can effectively visualise. A graph \(\mathcal{G} = (V, \mathcal{E})\), where \(V\) is the set of vertices or nodes and \(\mathcal{E}\) is the set of edges that connect these nodes and form the skeleton of the graph. Nodes correspond to the variables in the dataset, while edges represent connections between these variables. Edges can be directed, undirected, bidirected, or have no edge between two nodes.

A graph containing only one-way directed edges and no cycles is called a Directed Acyclic Graph (DAG). In a DAG, an edge direction represents a belief that there is a direct causal relation between two variables. For example, an edge \(X \rightarrow Y\) means that variable \(X\) causes \(Y\). An absence of an edge represents that there is no direct relation, and the variables are marginally independent, hence d-separated on the graph given the set \(\mathbf{S} = \{\emptyset\}\), according to the causal Markov property.

D-separation is a graphical criterion used to determine whether a set of variables is conditionally independent of another set, given a third set \((X \perp_{G} Y \mid \mathbf{S})\). We can infer d-separation statements from data, relying on the faithfulness assumption.
\[
X \perp\!\!\!\perp_P Y \mid \mathbf{S} \implies X \perp_G Y \mid \mathbf{S},
\] And use the global Markov property to generate the learned distribution.
\[
X \perp_G Y \mid S \implies X \perp\!\!\!\perp_P Y \mid S,
\]
Together, these properties establish a correspondence between the conditional independencies observed in the data and the d-separation statements in the graph.

Every DAG encodes a set of d-separation statements, and a DAG can be learned from such statements. However, it is not always possible to learn a unique DAG from the data. Instead, one can identify a Markov Equivalence Class (MEC) of DAGs that share the same skeleton and v-structures, which is represented by a Completed Partially Directed Acyclic Graph (CPDAG). CPDAGs contain both directed and undirected edges. The presence of an undirected edge in a CPDAG indicates a connection between two variables, but the causal direction cannot be determined due to insufficient information in the data.

A graph that contains all kinds of edges (directed, undirected, and bidirected) is called a mixed graph. A bidirected edge between nodes, e.g., \( X \leftrightarrow Y \), indicates the presence of a common unobserved (latent) confounder \( Z \) such that \( X \leftarrow Z \rightarrow Y \). A mixed graph is called ancestral if it contains no directed cycles or almost (partially) directed cycles, and for any undirected edge \( X - Y \), the nodes \( X \) and \( Y \) have no parents or spouses. A Directed Acyclic Graph (DAG) is a special case of an ancestral graph containing only directed edges and no cycles. The graph is called maximal if for any two non-adjacent nodes, there exists a set of vertices that m-separates them. M-separation is a generalisation of d-separation for more complex graphs such as Maximal Ancestral Graphs (MAGs). 

Similar to DAGs, it might be difficult to learn a MAG from data. However, we can extract a Markov Equivalence Class (MEC) of MAGs by learning a Partial Ancestral Graph (PAG). Markov equivalence rules for MAGs are more complex as for DAGs: a set of MAGs form a MEC if and only if, besides having the same skeleton and unshielded colliders, they also satisfy the condition if a path \( p \) is a discriminating path for a vertex \( V \) in both graphs, then \( V \) is a collider on \( p \) in one graph if and only if it is a collider on \( p \) in the other.
This additional rule ensures that discriminating paths are consistent across equivalent MAGs, maintaining the equivalence beyond what skeletons and unshielded colliders can capture. PAGs represent these equivalence classes. They encode all possible causal structures consistent with the observed data and latent confounding.

 Various algorithms can be employed to uncover these relationships from observational data. m-separations are learned from data using conditional independence testing. In this particular report, we will focus on constraint-based methods that utilise conditional independence tests for discrete data to understand both the presence and direction of a causal path in the chosen dataset. We aim to to apply the FCI algorithm and retrieve a PAG that describes causal relations between the environment and bullying in teenagers aged 11 to 16 years, based on HBSC data, and to interpret and analyze the results of a selected algorithm while considering possible improvements, advantages, and drawbacks of the chosen approach.




% A set of Markov equivalent graphs forms the Markov Equivalence Class. Two graphs are Markov Equivalent if and only if they have the same skeleton and v-structure. The skeleton of a graph $\mathcal{G}$ is the set of edges $\mathcal{E}$.

% [Introduce causal discovery and its relevance in analyzing real-world observational data such as HBSC. Briefly mention the motivation for exploring adolescent health behaviors using a causal lens, and the expected outcomes of your analysis.]
\end{document}