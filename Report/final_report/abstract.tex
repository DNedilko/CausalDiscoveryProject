\documentclass[main.tex]{subfiles}
\begin{document}
In this study, we explore the causes of bullying and cyberbullying in the digital age, focusing on adolescents in Ukraine. Using constraint-based causal discovery methods, specifically the Fast Causal Inference (FCI) algorithm, we analyse data from the 2018 Health Behaviour in School-aged Children (HBSC) self-report questionnaire. These methods rely on conditional independence tests for discrete data to recover Partial Ancestral Graphs (PAGs) and uncover both the presence and direction of causal pathways. Our findings highlight \textit{beenbullied} as a central node in the causal structure, influenced by traditional and cyber forms of bullying, and linked to school pressure, peer acceptance, and family communication. Contrary to previous notions that digital technology alone drives bullying behaviours. Instead, our results suggest that bullying is deeply rooted in broader social dynamics, signalling that effective prevention must shift its focus to influence school climate, peer relationships, and family support structures.
\end{document}