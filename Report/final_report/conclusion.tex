\documentclass[main.tex]{subfiles}
\begin{document}
\section{Conclusion}

This study applied the FCI algorithm to explore the causal structure underlying bullying and cyberbullying among Ukrainian adolescents using cross-sectional data from the 2018 HBSC survey. By leveraging conditional independence tests and accounting for potential latent confounding, we produced Partial Ancestral Graphs that revealed consistent patterns of interaction among behavioural, social, and academic factors.

The results emphasise the centrality of peer victimisation within a network of aggression, academic stress, and social acceptance. Contrary to common expectations, digital device usage did not emerge as a dominant causal factor, suggesting that cyberbullying reflects existing offline dynamics rather than being driven primarily by technology. This finding underscores the importance of addressing underlying social and emotional determinants rather than focusing solely on digital access or media exposure.

Methodologically, the analysis demonstrates the sensitivity of constraint-based causal discovery to choices of test statistic and significance threshold. An $\alpha$ level of 0.01 combined with the G$^2$ test yielded the most interpretable and stable structures, balancing false discovery control with model complexity.

While the observational and self-reported aspects of the dataset restrict the ability to make strong causal claims, the structural insights gathered are significant and typically consistent with theoretical expectations. Nevertheless, these results should be interpreted cautiously and verified by field experts. Future research could enhance this analysis by integrating longitudinal data, various psychosocial variables, and different causal discovery techniques to validate and fine-tune the inferred relationships.

\end{document}