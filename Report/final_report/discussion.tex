\documentclass[main.tex]{subfiles}
\begin{document}
\section{Discussion}

This study used the Fast Causal Inference (FCI) algorithm to explore causal relationships in adolescent bullying behaviour based on the 2018 HBSC dataset. The output is a Partial Ancestral Graph (PAG), a representation that encodes causal and ancestral relations under the assumption of potential latent confounding and selection bias. As such, the output does not fully orient all edges but provides a consistent representation of the underlying equivalence class of causal structures. This enables us to interpret the results carefully, suggesting strong dependencies yet not always clear cause-and-effect directionality.

Across multiple settings, \textit{beenbullied} consistently appeared as a central node, connected to \textit{cbulliedothers}, \textit{fight12m}, and \textit{studaccept}. These findings imply a connection between victimisation, aggressive conduct, and social integration. Interestingly, direct measures of digital device usage did not serve as significant causes, which contrasts with prevalent assumptions regarding smartphones' direct influence on cyberbullying. Instead, the results endorse the perspective that wider psychosocial factors, such as peer relationships and academic experiences, influence bullying behaviours.

The causal structure varied significantly with the choice of the conditional independence (CI) test and the significance level $\alpha$. Graphs generated using the G$^2$ test were generally sparser and more interpretable than those from $\chi^2$, which tended to produce denser structures. Lower $\alpha$ thresholds (e.g., $0.001$) yielded conservative graphs, reducing potential false positives, while higher thresholds increased edge count at the cost of interpretability. An $\alpha$ level of 0.01 with the G$^2$ test was found to offer a good trade-off between complexity and robustness.

The assumptions behind FCI, particularly the Causal Markov Condition and Faithfulness, are essential for valid inference. Although these assumptions are standard in causal discovery, their applicability to social survey data is not guaranteed. Furthermore, because of the observational and cross-sectional characteristics of the HBSC data, along with the discretisation necessary for the algorithm, causal claims ought to be understood in a structural context.  The outcome causal paths should be carefully reviewed and interpreted in conjunction with background knowledge and broader behavioural analysis knowledge.

Bidirected edges in the PAG indicate the presence of latent confounding. These likely correspond to unmeasured psychological traits, familial factors, or environmental variables not included in the HBSC survey. Their existence emphasises the strength of FCI in handling partially observed systems and highlights the need for further data collection in future studies.

In general, the inferred relationships are structurally sound and largely align with current psychological and sociological research. Importantly, the limited causal influence of technology-related factors challenges prevailing assumptions held by the general public and points to avenues for future research.

\end{document}
